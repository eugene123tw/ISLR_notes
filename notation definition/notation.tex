\input cinput.tex    \documentclass[12pt,a4paper]{article}%
\usepackage{makeidx}
\makeindex
\usepackage{bm}
\usepackage{framed} % Easier way to use Framebox
\usepackage{pdfpages} % Import PDF in latex document
\usepackage{fancybox}
\usepackage{ulem} % for text strikethrough
\usepackage{tikz}
\usepackage{listings}
\usepackage{slashbox}
\usepackage{array}
\usepackage{enumitem}
\usepackage{amsmath, amssymb, amsthm}  % For mathematical symbols
\usepackage{rotating, booktabs}  % For table-rotating
\usepackage{longtable,booktabs}
\usepackage{wallpaper}  % For watermark
\usepackage{colortbl,color}
\usepackage{xcolor}
\usepackage{graphicx,psfrag}
\usepackage{tabularx,array}
\usepackage{booktabs}
\usepackage{multirow}
\usepackage{multicol}
\usepackage[subfigure]{tocloft}
\usepackage[tight]{subfigure}
\usepackage{float,booktabs,threeparttable}
\usepackage{caption}
\usepackage{menukeys}
\usepackage{pst-tree}
\usepackage{longtable}
\usepackage{appendix}
\usepackage{adjustbox}
\usepackage{pdfpages}

\def\se{{\rm se}}
%\newcommand{\red}{\color{red}}
\linespread{1.5}  % The linespread is 1.5.

% Numbered theorems, definitions, algorithm and lemmas ======================================================================
\newtheorem{thm}{Theorem}  % Define new theorem.
\newtheorem{alg}{Algorithm}[section]  % Define new algorithm.
\newtheorem{definition}{Definition}
% ===========================================================================================================================

% For writing pseudo code ======================================================================
\usepackage{algorithm}% http://ctan.org/pkg/algorithms
\usepackage{algpseudocode}% http://ctan.org/pkg/algorithmicx
% ===========================================================================================================================

\theoremstyle{definition}
\theoremstyle{plain}
\setcounter{secnumdepth}{5}


\renewcommand{\contentsname}{Table of Contents}
\renewcommand{\listfigurename}{List of Figures}
\renewcommand{\listtablename}{List of Tables}
\renewcommand{\figurename}{\footnotesize Figure}
\renewcommand{\tablename}{\footnotesize Table}
\newcommand{\loflabel}{Figure}
\newcommand{\lotlabel}{Table}
\setlength{\abovecaptionskip}{0pt}

%\renewcommand{\cftsecpresnum}{Chapter }

\renewcommand{\cftsecnumwidth}{7em}
%\renewcommand{\thesection}{{\ctxfbb Chapter}~~ \arabic{section}}
%\renewcommand{\thesubsection}{\arabic{section}.\arabic{subsection}}
%\renewcommand{\thesubsubsection}{\arabic{section}.\arabic{subsection}.\arabic{subsubsection}}
\renewcommand{\appendixpagename}{\LargeAppendix} % \ctxfb
\renewcommand{\arraystretch}{1.2}

\usepackage{appendix}

\def\oo{\nolinebreak[4]\hspace{.3em}\raise.7ex\hbox{{\MaQ\cH1}}\hspace{0.3em}}
\def\pp{\nolinebreak[4]\hspace{.3em}\raise.8ex\hbox{,}\hspace{0.3em}}
\def\dd{\nolinebreak[4]\hspace{.3em}\raise.8ex\hbox{{\MaQ\cH2}}\hspace{0.3em}}
\def\kk{\nolinebreak[4]\hspace{.3em}\raise.3ex\hbox{;}\hspace{0.3em}}
\def\mm{\nolinebreak[4]\hspace{.3em}\raise.3ex\hbox{:}\hspace{0.3em}}
\def\aa{\nolinebreak[4]\hspace{.3em}\raise.3ex\hbox{?}\hspace{0.3em}}

%%%%%%%%%%

%%%%%%%%%%
\usepackage[refpage]{nomencl}
\renewcommand{\nomname}{Notations}
\renewcommand*{\pagedeclaration}[1]{\unskip\dotfill\hyperpage{#1}}
\newcommand\independent{\protect\mathpalette{\protect\independenT}{\perp}}
\def\independenT#1#2{\mathrel{\rlap{$#1#2$}\mkern2mu{#1#2}}}
\makenomenclature
\usepackage{makeidx}
\makeindex
%%%%%%%%%%%%
%\let\clipbox\relax

%%%%%%%%%%%%
%\ctxfdef{\section}{\ctxfbb}
%\ctxfdef{\subsection}{\ctxfbb} %\ctxfr

\def\tb#1#2{\mathop{#1\vphantom{\sum}}\limits_{\displaystyle #2}}

\newtheorem{lma}{\textbf{Lemma}}

% ======================== Set length ========================
\setlength{\columnsep}{2.4cm}
\setlength\parindent{0pt}
\textheight=22cm
\textwidth=16.5cm
\hoffset=-1cm
% ============================================================


% =============================
% Equation numbering
\numberwithin{equation}{section}
% =============================

\begin{document}
\section{Notations}

$\mathbf{X}$ denote a $n \times p$ matrix whose $(i,j)$th element is $x_{ij}$ That is,
\begin{gather}
\mathbf{X}_{n \times p} = \begin{pmatrix}
  x_{11} & x_{12} & \cdots & x_{1p}  \\
  x_{21} & x_{22} & \cdots & x_{2p}  \\
  \vdots & \vdots & \ddots & \vdots  \\
  x_{n1} & x_{n2} & \cdots & x_{np}
 \end{pmatrix}
\end{gather}

Transpose of $\mathbf{X}$,
\begin{gather}
\mathbf{X}^{T}_{p \times n} = \begin{pmatrix}
  x_{11} & x_{21} & \cdots & x_{n1}  \\
  x_{12} & x_{22} & \cdots & x_{n2}  \\
  \vdots & \vdots & \ddots & \vdots  \\
  x_{1p} & x_{2p} & \cdots & x_{np}
 \end{pmatrix}
\end{gather}

$x_{i}$ is a vector of length $p$ containing the $p$ variable for the $i$th observation. That is,
\begin{gather}
x_{i} = \begin{pmatrix}
  x_{i1} \\
  x_{i2} \\
  \vdots \\
  x_{ip} 
 \end{pmatrix}
\end{gather}

Transpose of $x_{i}$
\begin{gather}
x_{i}^{T} = \begin{pmatrix}
  x_{i1} & x_{i2} & \cdots & x_{ip}
 \end{pmatrix}
\end{gather}
~\\

The columns of $\mathbf{X}$, which we write as $\mathbf{x}_{1},\mathbf{x}_{2},\dots,\mathbf{x}_{p}$.
Each is a vector of length $n$. That is,
\begin{gather}
\mathbf{x}_{j} = \begin{pmatrix}
  x_{1j} \\
  x_{2j} \\
  \vdots \\
  x_{nj} 
 \end{pmatrix}
\end{gather}

Using the notation above, the matrix $\mathbf{X}$ can be written as
\begin{gather}
\mathbf{X} =  (\mathbf{x}_{1} ~~ \mathbf{x}_{2} ~~  \cdots ~~  \mathbf{x}_{p})
\end{gather}

Or, 
\begin{gather}
\mathbf{X} =  \begin{pmatrix}
  x_{1}^{T} \\
  x_{2}^{T} \\
  \vdots \\
  x_{n}^{T}
 \end{pmatrix}
\end{gather}

$y_{i}$ denote the $i$th response. $\mathbf{y}$ for all $n$ response in vector form as,
\begin{gather}
\mathbf{y} =  \begin{pmatrix}
  y_{1} \\
  y_{2} \\
  \vdots \\
  y_{n}
 \end{pmatrix}
\end{gather}



\end{document}
